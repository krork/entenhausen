\documentclass{article}
\usepackage{amsmath}
\usepackage[utf8]{inputenc}
\usepackage{float}
\usepackage{epsfig,graphicx}
\usepackage{xcolor,import}
\usepackage[german]{babel}
\usepackage{textcomp}
\usepackage{mathtools}

\begin{document}
\thispagestyle{empty}
			\begin{center}
			\Large{Fakultät für Physik}\\
			\end{center}
\begin{verbatim}


\end{verbatim}
							%Eintrag des Wintersemesters
			\begin{center}
			\textbf{\LARGE WINTERSEMESTER 2014/15}
			\end{center}
\begin{verbatim}


\end{verbatim}
			\begin{center}
			\textbf{\LARGE{Physikalisches Praktikum 1}}
			\end{center}
\begin{verbatim}




\end{verbatim}

			\begin{center}
			\textbf{\LARGE{PROTOKOLL}}
			\end{center}
			
\begin{verbatim}





\end{verbatim}

			\begin{flushleft}
			\textbf{\Large{Experiment (Nr., Titel):}}\\
							%Experiment Nr. und Titel statt den Punkten eintragen
			\LARGE{3. Elastizität / Trägheitsmoment}	
			\end{flushleft}

\begin{verbatim}

\end{verbatim}	
							%Eintragen des Abgabedatums, oder des Erstelldatums des Protokolls
			\begin{flushleft}
			\textbf{\Large{Datum:}} \Large{31.10.2014}
			\end{flushleft}
			
\begin{verbatim}
\end{verbatim}
							%Namen der Protokollschreiber
		\begin{flushleft}
			\textbf{\Large{Namen:}} \Large{Veronika Bachleitner, Erik Grafendorfer}
			\end{flushleft}

\begin{verbatim}


\end{verbatim}
							%Kurstag und Gruppennummer, zb. Fr/5
			\begin{flushleft}
			\textbf{\Large{Kurstag/Gruppe:}} \Large{Fr/1}
			\end{flushleft}

\begin{verbatim}






\end{verbatim}
							%Name des Betreuers, das Praktikum betreute.
			\begin{flushleft}
			\LARGE{\textbf{Betreuer:}}	\Large{SETMAN}	
			\end{flushleft}
			
\newpage
\section{Spannungs-Dehnungs-Kurve von Aluminium}
\subsection{Aufgabenstellung}
Wir messen erst die Dicke eines Aluminiumdrahtes, dann spannen wir ihn zwischen die Balken eines Materialprüfgerätes, dehnen ihn in gleichen Zeitabschnitten gleiche kleine Längen und messen mit einem von uns kalibrierten Messverstärker die Kraft, die auf ihn wirkt. Aus der Kraft und der Dicke des Drahtes berechnen wir die Spannung, die auf ihn wirkt, vergleichen sie mit der Dehnung, die er erfuhr. Wir bestimmen so das Verhältnis der Spannung und der Dehnung des Drahtes, den E-Modul und versuchen seine Streckgrenze auf verschiedene Arten zu ermitteln.
\subsection{Grundlagen}
Wenn Kräfte auf Festkörper wirken, gibt es zwei physikalisch völlig unterschiedliche Arten der Verformung, die sie erleiden: Die \textit{elastische} und die \textit{plastische} Verformung. Die elastische Verformung basiert auf Kräften zwischen den Atomen selbst, während die plastische auf Deformierungen im Kristallgitter des Körpers beruht. \\
\\Die Spannung $\sigma$ auf einen Körper ist die auf ihn wirkende Kraft pro der Fläche, auf der sie wirkt:  $$ \sigma = \frac{F}{A} $$ \\
Die Dehnung  $\epsilon$ ist das Verhältnis aus der Längenänderung $\Delta$l und seiner ganzen Länge l, die der Körper bei der Dehnug erfährt: \\
$$ \epsilon= \frac{\Delta l}{l} $$ \\
 Wir überlegen: Während einer elastischen Verformung des Drahtes bleibt das Verhältnis von Spannung und Dehnung annähernd konstant, weil es keine qualitativen Änderungen in seiner Struktur gibt - darum kehrt ein Körper nach einer elastischen Verformung auch in seinen Ausgangszustand zurück. Beginnt allerdings die plastische Verformung, gibt es kein zurück - die Kristallstruktur des armen Körpers wird dauerhaft verändert, seine Dehnung fällt leichter weil seine Integrität zerstört wird. Wir wollen den Übergang von elastischer zu plastischer Verformung genau ermitteln, indem wir die Spannung $\sigma$ pro Dehnung $\epsilon$ darstellen und untersuchen, wo das Verhältnis konstant ist und wo plötzlich mehr Dehnung bei gleich anwachsender Spannung gemessen wird. Den Quotienten aus der Spannung und der Dehnung nennt man den \textit{E-Modul E}:

$$ E=\frac{\sigma}{\epsilon}$$

Einheiten der Größen:
$$[\sigma]=\frac{N}{m^2}$$
$$[E]=\frac{N}{m^2}$$
\subsection{Versuchsaufbau und Methoden}
Wir verwenden Prüfmaschine der Fa. INSTRON zur Dehnung des Drahts. Diese Maschine besitzt einen Ausgang, der eine Spannung relativ zur auf den Draht gemessenen Kraft ausgibt und mittels eines Messverstärkers verstärkt wird. Dieses Signal wird dann mittels eines Analog/Digitalwandlers der Firma National Instruments an den Messcomputer gesendet. Parallel zum Analogausgang des Verstärkers wird ein Voltmeter zur Kalibrierung des Messverstärkers geschalten.
Wir erhalten $\sigma$ durch Division der gemessenen Kraft durch den gemessenen Probenquerschnitt q:
$$ \sigma = \frac{F}{q}$$
$\epsilon$ ist das Verhältnis der Längenänderung, die die Maschine verursacht, v$\cdot$t, zur ursprünglichen Einspannlänge der Probe.
$$ \epsilon = \frac{v\cdot t}{L_0} \text{hier} = \frac{0.5\frac{mm}{min}}{100mm} = \frac{8.\bar{3}\cdot10^{-3}\frac{mm}{s}}{100mm}$$
\subsection{Ergebnisse}
Literaturwert [Tipler] zum E-Modul von Aluminum: 70 kN/mm²
\subsection{Diskussion}
\section{Bestimmung des Schubmoduls von Aluminium mit dem Torsionspendel}
\subsection{Aufgabenstellung}
Wir befestigen zwei Gewichte an den Enden einer Querstange, hängen diese an ihrer Mitte an einen Aluminiumdraht und versetzen die Stange ihn Rotation. Aus ihrer Schwingungsdauer berechnen wir dann das Schubmodul, auch Torsionsmodul genannt, des Aluminiumdrahts.
\subsection{Grundlagen}

\subsection{Versuchsaufbau und Methoden}
\subsection{Ergebnisse}
Literaturwert [Wikipedia] zum Schubmodul von Aluminium: 25.5 kN/mm² = GPa
\subsection{Diskussion}
\section{Physikalisches Pendel}
\subsection{Aufgabenstellung}
Wir führen Messungen mit einem Fahrradpendel durch. Dazu messen wir die Schwingungsdauer in Abhängigkeit des Auslenkwinkels über einen Bereich von 10° bis 100° dar und verarbeiten und diskutieren anschließend unsere Messwerte. 
Daraus berechnen wir uns auch weitere Ergebnisse wie die Winkelrichtgröße des Pendels und die Trägheitsmomente von Pendel, Zusatzmasse und Rad. 

\subsection{Grundlagen}
Auslenkwinkel $\phi(t)$: [$\phi(t)$]=1(rad)\\
Zeit t: $[t]=s$\\
Masse m: $[m]=kg$\\
Länge l: $[l]=m$\\
Periodendauer T: $[T]=s$\\
Winkelbeschleunigung $\ddot{\phi}$: [$\ddot{\phi}$]=$s^{-2}$\\
Winkelrichtgröße D: $[D]=kg \frac{m^2}{s^2}$\\
Erdbeschleunigung: $g=9.81ms^{-2}$\\
\\
\textbf{Mathematisches Pendel}\\
Ein mathematische Pendel ist ein idealisiertes Objekt und besteht aus einem masselosen Faden und einer im Schwerpunkt konzentrierten Masse. Reibungs- und Strömungswiderstände werden vernachlässigt, sodass eine konstante Schwingungsdauer $T_0$ angenommen werden kann.\\
\\
Bewegungsgleichung:
\begin{equation*}
\ddot{\phi}(t)+\frac{g}{l}sin(\phi(t))=0
\end{equation*}\\
Lösung für kleine Winkel:
\begin{equation*}
\phi(t)=\phi_0cos(\sqrt{\frac{g}{l}}t+\Phi)
\end{equation*}
wobei $\phi_0$ der maximale Auslenkungswinkel und $\Phi$ die Anfangsphase bei t=0 ist. \\
\\
Periodendauer:
\begin{equation*}
T_0 = \frac{2\pi}{\omega} = 2\pi\sqrt{\frac{l}{g}}
\end{equation*}\\
\\
\textbf{Physikalisches Pendel}\\
Im Gegensatz zum mathematischen Pendel, schwingt beim physikalischen Pendel ein Körper um eine Achse, die \textit{nicht} durch den Schwerpunkt verläuft.\\
Die Bewegungsgleichung muss jetzt anders bestimmt werden:
\begin{equation*}
J\ddot{\phi(t)} + D sin(\phi(t)) = 0
\end{equation*}
wobei die \textbf{Winkelrichtgröße D}=$mgl$\\
\\
Schwingungsdauer:
\begin{equation*}
T_0=\frac{2\pi}{\omega}=2\pi\sqrt{\frac{J}{D}}
\end{equation*}\\
\\
\textbf{Steiner'scher Satz}\\
\begin{equation*}
J = J_S + m_{ges} d^2
\end{equation*}
wobei $J_S$ das Trägheitsmoment bezüglich einer Achse um den Schwerpunkt, J das Trägheitsmoment bezüglich einer parallelen Achse im Abstand d.
\\
\textbf{Fahrradpendel}\\
Ein in Wirklichkeit realisiertes Pendel kann folgendermaßen aussehen:
%\includegraphics[scale=0.1,angle=-90]{fahrradpendel.eps}
Wir sehen hier, dass es sich hierbei um ein Speichenrad mit zylindrischer Zusatzmasse handelt.\\
\\
Winkelrichtgröße:
\begin{equation*}
D=m_Z g l_{AS}
\end{equation*}
wobei $l_{AS}=R_F + \frac{h_Z}{2}$, $R_F$ der Radius der Felge, $h_Z$ die Höhe der Zusatzmasse.\\
\\
Trägheitsmoment:
\begin{equation*}
J = J_Z + J_{Rad} = D (\frac{T_0}{2/pi})^2
\end{equation*}
\\
wobei das Trägheitsmoment des Zylinders:
\begin{equation*}
J_Z = m_Z(\frac{1}{4}r^2_Z + \frac{1}{12}h^2_Z+l^2_{AS}
\end{equation*}
\\
Das Trägheitsmoment des Rades wird rückschließend bestimmt durch Nutzung der Gleichung für das gesamte Trägeheitsmoment.

\subsection{Versuchsaufbau und Methoden}
Als Messgerät wird CASSY-Lab2 verwendet.\\
\\
\subsection{Ergebnisse}
\textbf{Schwingungsdauer T}\\
T als Funktion von $\phi_0$: \\
Schwingungsdauer für sehr kleine Auslenkungen:\\
\\
\textbf{Winkelrichtgröße}\\
\\
\textbf{Trägheitsmoment des Pendels}\\
\\
\textbf{Trägheitsmoment der Zusatzmasse}\\
\\
\textbf{Trägheitsmoment des Rades}\\
\\
\textbf{Fehlerrechnung}\\
Sensordaten:\\
Frequenzauflösung: 0.001Hz\\
Winkelauflösung: 0.18°\\
Wegauflösung: 0.08mm\\
Zeitauflösung: 0.001s\\
$\boxed{Endergebnis}$ %command merken
\subsection{Diskussion}
Bis zu welchen Auslenkungswinkeln stimmt die Ausgleichskurve mit der experimentellen Kurve überein? 

\end{document}
