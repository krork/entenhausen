\documentclass{article}
\usepackage{amsmath}
\usepackage[utf8]{inputenc}
\usepackage{float}
\usepackage{epsfig,graphicx}
\usepackage{xcolor,import}
\usepackage[german]{babel}
\usepackage{textcomp}
\usepackage{mathtools}

\begin{document}
\thispagestyle{empty}
			\begin{center}
			\Large{Fakultät für Physik}\\
			\end{center}
\begin{verbatim}


\end{verbatim}
							%Eintrag des Wintersemesters
			\begin{center}
			\textbf{\LARGE WINTERSEMESTER 2014/15}
			\end{center}
\begin{verbatim}


\end{verbatim}
			\begin{center}
			\textbf{\LARGE{Physikalisches Praktikum 1}}
			\end{center}
\begin{verbatim}




\end{verbatim}

			\begin{center}
			\textbf{\LARGE{PROTOKOLL}}
			\end{center}
			
\begin{verbatim}





\end{verbatim}

			\begin{flushleft}
			\textbf{\Large{Experiment (Nr., Titel):}}\\
							%Experiment Nr. und Titel statt den Punkten eintragen
			\LARGE{3. Elastizität / Trägheitsmoment}	
			\end{flushleft}

\begin{verbatim}

\end{verbatim}	
							%Eintragen des Abgabedatums, oder des Erstelldatums des Protokolls
			\begin{flushleft}
			\textbf{\Large{Datum:}} \Large{31.10.2014}
			\end{flushleft}
			
\begin{verbatim}
\end{verbatim}
							%Namen der Protokollschreiber
		\begin{flushleft}
			\textbf{\Large{Namen:}} \Large{Veronika Bachleitner, Erik Grafendorfer}
			\end{flushleft}

\begin{verbatim}


\end{verbatim}
							%Kurstag und Gruppennummer, zb. Fr/5
			\begin{flushleft}
			\textbf{\Large{Kurstag/Gruppe:}} \Large{Fr/1}
			\end{flushleft}

\begin{verbatim}






\end{verbatim}
							%Name des Betreuers, das Praktikum betreute.
			\begin{flushleft}
			\LARGE{\textbf{Betreuer:}}	\Large{SETMAN}	
			\end{flushleft}
			
\newpage
\section{Physikalisches Pendel}
\subsection{Aufgabenstellung}
Wir führen Messungen mit einem Fahrradpendel durch. Dazu messen wir die Schwingungsdauer in Abhängigkeit des Auslenkwinkels über einen Bereich von 10° bis 100° dar und verarbeiten und diskutieren anschließend unsere Messwerte. 
Daraus berechnen wir uns auch weitere Ergebnisse wie die Winkelrichtgröße des Pendels und die Trägheitsmomente von Pendel, Zusatzmasse und Rad. 

\subsection{Grundlagen}
Auslenkwinkel $\phi(t)$: [$\phi(t)$]=1(rad)\\
Zeit t: $[t]=s$\\
Masse m: $[m]=kg$\\
Länge l: $[l]=m$\\
Periodendauer T: $[T]=s$\\
Winkelbeschleunigung $\ddot{\phi}$: [$\ddot{\phi}$]=$s^{-2}$\\
Winkelrichtgröße D: $[D]=kg \frac{m^2}{s^2}$\\
Erdbeschleunigung: $g=9.81ms^{-2}$\\
\\
\textbf{Mathematisches Pendel}\\
Ein mathematische Pendel ist ein idealisiertes Objekt und besteht aus einem masselosen Faden und einer im Schwerpunkt konzentrierten Masse. Reibungs- und Strömungswiderstände werden vernachlässigt, sodass eine konstante Schwingungsdauer $T_0$ angenommen werden kann.\\
\\
Bewegungsgleichung:
\begin{equation*}
\ddot{\phi}(t)+\frac{g}{l}sin(\phi(t))=0
\end{equation*}\\
Lösung für kleine Winkel:
\begin{equation*}
\phi(t)=\phi_0cos(\sqrt{\frac{g}{l}}t+\Phi)
\end{equation*}
wobei $\phi_0$ der maximale Auslenkungswinkel und $\Phi$ die Anfangsphase bei t=0 ist. \\
\\
Periodendauer:
\begin{equation*}
T_0 = \frac{2\pi}{\omega} = 2\pi\sqrt{\frac{l}{g}}
\end{equation*}\\
\\
\textbf{Physikalisches Pendel}\\
Im Gegensatz zum mathematischen Pendel, schwingt beim physikalischen Pendel ein Körper um eine Achse, die \textit{nicht} durch den Schwerpunkt verläuft.\\
Die Bewegungsgleichung muss jetzt anders bestimmt werden:
\begin{equation*}
J\ddot{\phi(t)} + D sin(\phi(t)) = 0
\end{equation*}
wobei die \textbf{Winkelrichtgröße D}=$mgl$\\
\\
Schwingungsdauer:
\begin{equation*}
T_0=\frac{2\pi}{\omega}=2\pi\sqrt{\frac{J}{D}}
\end{equation*}\\
\\
\textbf{Steiner'scher Satz}\\
\begin{equation*}
J = J_S + m_{ges} d^2
\end{equation*}
wobei $J_S$ das Trägheitsmoment bezüglich einer Achse um den Schwerpunkt, J das Trägheitsmoment bezüglich einer parallelen Achse im Abstand d.
\\
\textbf{Fahrradpendel}\\
Ein in Wirklichkeit realisiertes Pendel kann folgendermaßen aussehen:
%\includegraphics[scale=0.1,angle=-90]{fahrradpendel.eps}
Wir sehen hier, dass es sich hierbei um ein Speichenrad mit zylindrischer Zusatzmasse handelt.\\
\\
Winkelrichtgröße:
\begin{equation*}
D=m_Z g l_{AS}
\end{equation*}
wobei $l_{AS}=R_F + \frac{h_Z}{2}$, $R_F$ der Radius der Felge, $h_Z$ die Höhe der Zusatzmasse.\\
\\
Trägheitsmoment:
\begin{equation*}
J = J_Z + J_{Rad} = D (\frac{T_0}{2/pi})^2
\end{equation*}
\\
wobei das Trägheitsmoment des Zylinders:
\begin{equation*}
J_Z = m_Z(\frac{1}{4}r^2_Z + \frac{1}{12}h^2_Z+l^2_{AS}
\end{equation*}
\\
Das Trägheitsmoment des Rades wird rückschließend bestimmt durch Nutzung der Gleichung für das gesamte Trägeheitsmoment.

\subsection{Versuchsaufbau und Methoden}
Als Messgerät wird CASSY-Lab2 verwendet.\\
\\
\subsection{Ergebnisse}
\textbf{Schwingungsdauer T}\\
T als Funktion von $\phi_0$: \\
Schwingungsdauer für sehr kleine Auslenkungen:\\
\\
\textbf{Winkelrichtgröße}\\
\\
\textbf{Trägheitsmoment des Pendels}\\
\\
\textbf{Trägheitsmoment der Zusatzmasse}\\
\\
\textbf{Trägheitsmoment des Rades}\\
\\
\textbf{Fehlerrechnung}\\
Sensordaten:\\
Frequenzauflösung: 0.001Hz\\
Winkelauflösung: 0.18°\\
Wegauflösung: 0.08mm\\
Zeitauflösung: 0.001s\\
$\boxed{Endergebnis}$ %command merken
\subsection{Diskussion}
Bis zu welchen Auslenkungswinkeln stimmt die Ausgleichskurve mit der experimentellen Kurve überein? 

\end{document}
