\documentclass{article}
\usepackage{amsmath}
\usepackage[utf8]{inputenc}
\usepackage{float}
\usepackage{epsfig,graphicx}
\usepackage{xcolor,import}
\usepackage[german]{babel}
\usepackage{textcomp}
\usepackage{mathtools}

\begin{document}
\thispagestyle{empty}
			\begin{center}
			\Large{Fakultät für Physik}\\
			\end{center}
\begin{verbatim}


\end{verbatim}
							%Eintrag des Wintersemesters
			\begin{center}
			\textbf{\LARGE WINTERSEMESTER 2014/15}
			\end{center}
\begin{verbatim}


\end{verbatim}
			\begin{center}
			\textbf{\LARGE{Physikalisches Praktikum 1}}
			\end{center}
\begin{verbatim}




\end{verbatim}

			\begin{center}
			\textbf{\LARGE{PROTOKOLL}}
			\end{center}
			
\begin{verbatim}





\end{verbatim}

			\begin{flushleft}
			\Large{\textbf{Experiment Nr. 6:} Geometrische Optik}\\
							%Experiment Nr. und Titel statt den Punkten eintragen
			\LARGE{}	
			\end{flushleft}

\begin{verbatim}

\end{verbatim}	
							%Eintragen des Abgabedatums, oder des Erstelldatums des Protokolls
			\begin{flushleft}
			\textbf{\Large{Datum:21.11.}} \Large{2014}
			\end{flushleft}
			
\begin{verbatim}
\end{verbatim}
							%Namen der Protokollschreiber
		\begin{flushleft}
			\textbf{\Large{Namen:}} \Large{Veronika Bachleitner, Erik Grafendorfer}
			\end{flushleft}

\begin{verbatim}


\end{verbatim}
							%Kurstag und Gruppennummer, zb. Fr/5
			\begin{flushleft}
			\textbf{\Large{Kurstag/Gruppe:}} \Large{Fr/1}
			\end{flushleft}

\begin{verbatim}






\end{verbatim}
							%Name des Betreuers, das Praktikum betreute.
			\begin{flushleft}
			\LARGE{\textbf{Betreuer:}}	\Large{Stana}	
			\end{flushleft}
\newpage	

\section{Brennweite von Linsen}

\subsection{Aufgabenstellung}
Wir bestimmen die Brennweiten und die Brechkraft von einer Konvexlinse und einer Konkavlinse.
\subsection{Grundlagen}
Wir verwenden die \textbf{Abbildungsgleichung für Linsen:}
\begin{equation}
\label{linsengleichung}
\frac{1}{f}=\frac{1}{g}+\frac{1}{b}
\end{equation}
$\frac{1}{f}$ ist die Brechkraft der Linse und wird in Dioptrien ([1/m]) angegeben. f ist die Brennweite, g die Gegenstandsweite, b die Bildweite.\\ 
\\
Wir verwenden auch das \textbf{Besselverfahren}, das sich durch eine Distanz zwischen Schirm und Gegenstand von einem Minimum vom vierfachen der Brennweite der zu untersuchenden Linse charakterisiert. Durch es ergibt sich die Brennweite f zu 
\begin{equation}
\label{bessel}
f=\frac{1}{4}(e-\frac{d^2}{e})
\end{equation}
wobei e die Distanz zwischen Schirm und Gegenstand und d die Distanz zwischen den beiden Positionen der Linse bezeichnet, an denen sie ein scharfes Bild liefert.
\subsection{Versuchsaufbau und Methoden}
Wir haben eine Schiene mit verschiebbaren Linsen. Darauf ist eine Lichtquelle, vor der wir einen Gegenstand platzieren. Dieser wird durch die Linsen auf einen verschiebbaren Schirm abgebildet. Wir verschieben den Schirm und die Linsen auf Positionen auf denen sich scharfe Bilder ergeben und messen ihre Distanzen; aus diesen ermitteln wir die Eigenschaften der Linsen.
\subsection{Durchführung}
Bei der Durchführung gelang es uns trotz Hilfe nicht, ein scharfes Bild mit der Konkavlinse zu erzeugen. Wir müssen diesen Punkt daher streichen	, da sich keine Ergebnisse präsentierten, die im Reich der Vernunft anzusiedeln wären.
\subsection{Ergebnisse}
\subsubsection*{Brennweite einer Konvexlinse}
\paragraph{Bei einer Gegenstandsweite}

Wir messen 5 Mal die Bildweite b bei fixer Gegenstandsweite und berechnen den Mittelwert sowie seine Standardabweichung.

$$\bar{b}=(0.65 \pm 0.005)m$$

Da seine Unsicherheit, die Standardabweichung, 5 mal größer ist als die Unsicherheit der Gegenstandsweite von $\pm 0.001m$, verwenden wir nur diese bei der Berechnung der Unsicherheit der Brechkraft mit dem Gaußschen Fehlerfortpflanzungsgesetz. \\
Mittels (\ref{linsengleichung}) erhalten wir:\\
\boxed{$$f=(0.0971 \pm 0.007) m$$}\\
\boxed{$$\frac{1}{f}=(10.30 \pm 0.7) Dioptrien$$}
\paragraph{Besselverfahren}
Wir messen 5 Paare von Distanzen zwischen den beiden scharfen Positionen der Linse, d, und zwischen Schirm und Gegenstand, e. Daraus berechnen wir mittels (\ref{bessel}) die 5 zugehörigen Brennweiten und ermitteln ihren Mittelwert.\\
\boxed{$$\bar{f}=(0.0963\pm 0.002)m$$}\\
\boxed{$$\frac{1}{\bar{f}}=(10.4 \pm 0.3) Dioptrien $$}
\subsubsection*{Brennweite einer Konkavlinse}
Mit Konvexlinse g1 und b1 festsetzen. Abstand d der beiden Linsen. \\
Abstand vergrößern, zweite Bildweite b2 messen und Gegenstandsweiter g2 durch g2=-(b1-d) bestimmen.\\
in linsengleichung einsetzen

\subsubsection*{Brechkraft der Linsen}
\subsection{Diskussion}

\section{Linsenfehler}

\subsection{Aufgabenstellung}
Wir bestimmen die sphärische Aberration einer dicken Linse in den Durchgangsrichtungen plan-konvex und konvex-plan.
\subsection{Grundlagen}
\subsection{Versuchsaufbau und Methoden}
\subsection{Durchführung}
\subsection{Ergebnisse}
plan-konvex:\\
Die Hauptebene der Linse befindet sich im Punkt: $(0.615 \pm 0.005)mm$\\
3 Brennpunkte:\\
Enge Schlitze: $(0.748 \pm 0.001)mm$\\
Mittlere Schlitze: $(0.736 \pm 0.001)mm$\\
Weite Schlitze: $(0.705 \pm 0.001)$\\
\\
konvex-plan:\\
Die Hauptebene der Linse befindet sich im Punkt: $(0.615 \pm 0.002)mm$\\
3 Brennpunkte:
Enge Schlitze: $(0.695 \pm 0.001)mm$\\
Mittlere Schlitze: $(0.689 \pm 0.001)mm$\\
Weite Schlitze: $(0.683 \pm 0.001)$\\
\subsection{Diskussion}
Aus den Messungen zeigt sich, dass die Durchgangsrichtung konvex-plan weniger Fehler aufweist. Dies liegt daran, dass das Licht in diesem Fall zweimal gebrochen wird: Einmal durch die konvexe, einmal durch die plane Ebene. \\
In die andere Richtung ist das nicht so, da das Licht im Anfang durch die plane Ebene geht und hier nicht gebrochen wird. Dadurch stellt sich ein größerer Fehler aufgrund der konvexen Ebene ein. \\

EINFLUSS ANDERER LINSENFEHLER DISKUTIEREN?

\section{Mikroskop}

\subsection{Aufgabenstellung}
Wir bauen ein Mikroskop mithilfe zweier Konvexlinsen kleiner Brennweiten. Wir messen die Gesamtvergrößerung mithilfe einer Strichplatte als Gegenstand und testen die Abhängigkeit der Vergrößerung von der Tubuslänge. \\
Außerdem messen wir die Dicke eines Haares mit unserem Aufbau.
\subsection{Grundlagen}
Berechnung der Tubuslänge aus den Brennweiten und des Abstands der Linsen: $t=d-f_{Ob}-f_{Ok}$\\
Berechnung der Vergrößerung: $V_M=\frac{t_1}{f_{Ob}}\frac{s_0}{f_{Ok}}$
\subsection{Versuchsaufbau und Methoden}
\subsection{Durchführung}
\subsection{Ergebnisse}
Brennweite der als Objektiv verwendeten Linse: $f_{Ob}=25mm$\\
Brennweite der als Okular verwendeten Linse: $f_{Ok}=40mm$\\
Abstände der Linsen: $d_1=(93 \pm 1)mm$, $d_2=(77 \pm 1)mm$, $d_3=(85 \pm 1)mm$\\
Daraus die Tubuslängen: $t_1=(28 \pm 1)mm$, $t_2=(12 \pm 1)mm$, $t_3=(20 \pm 1)mm$ \\
Berechnete Vergrößerung: 
\begin{gather*}
V_{M_1}=7\\
V_{M_2}=3\\
V_{M_3}=5
\end{gather*}
\\
Gemessene Vergrößerung:\\
Für die erste Tubuslänge entsprechen 5mm 20 Einheiten auf der Skala, die 10mm/200 Striche zählt. Daraus ergibt sich eine Vergrößerung: $$\boxed{V_{M_1}=5}$$
Für die zweite Tubuslänge entsprechen 5mm 30 Einheiten auf der Skala. Daraus ergibt sich eine Vergrößerung: $$\boxed{V_{M_2}=3.3}$$
Für die dritte Tubuslänge entsprechen 5mm 23 Einheiten auf der Skala. Daraus ergibt sich eine Vergrößerung: $$\boxed{V_{M_3}=4.3}$$
\\
UNSICHERHEITEN BEI DER MESSUNG??
\\
Messung der Dicke eines Haares von Erik Grafendorfer:\\
Das Haar ist 2 Einheiten dick: Dies entspricht bei der Vergrößerung $V_{M_3}$ einer Dicke von: 
$$\boxed{d=(0.1 \pm 0.05)mm}$$
\subsection{Diskussion}

\section{Fernrohr}
\subsection{Aufgabenstellung}
Wir bauen ein Keplersches und ein Galileisches Fernrohr und bestimmen die erreichte Fernrohrvergrößerung.
\subsection{Grundlagen}
Ein Fernrohr besteht aus einem Objektiv und einem Okular, die in einem bestimmten Abstand voneinander positioniert sind.\\
Ein Keplersches (astronomisches) Fernrohr besteht aus zwei Sammellinsen. Dadurch entsteht ein umgekehrtes Bild. Angenehmer ist das Galileische (holländische) Fernrohr, dass das Bild durch Verwendung einer Sammel- und einer Zerstreuungslinse "richtig herum" darstellt.\\
\\
Die Fernrohrvergrößerung wird mit folgender Formel berechnet: 
$$V_F=\frac{f_{Ob}}{f_{Ok}}$$
\subsection{Ergebnisse}
\textbf{Keplersches Fernrohr:}
Brennweite der als Objektiv verwendeten Linse: $f_{Ob_K}=160mm$\\
Brennweite der als Okular verwendeten Linse: $f_{Ok_K}=120mm$\\
Berechnete Vergrößerung: 1.3\\
Beobachtete Vergrößerung: $1.5 \pm 0.5$\\
\\
\textbf{Galileisches Fernrohr:}
Brennweite der als Objektiv verwendeten Linse: $f_{Ob_G}=160mm$\\
Brennweite der als Okular verwendeten Linse: $f_{Ok_G}=50mm$\\
Berechnete Vergrößerung: 3.2\\
Beobachtete Vergrößerung: $3.0 \pm 0.5$\\

\subsection{Diskussion}
Die Ergebnisse passen sehr gut zu den berechneten Werten.\\
Bei der Beobachtung selbst haben wir aber zuerst das Problem, dass die Experimentatoren verschiedene Vergrößerungen wahrnehmen: Im ersten Fall eine Vergrößerung von 2.5 und eine von 1.5. Dieses Mysterium klärt sich durch die Beobachtung auf, dass ersterer selbst eine sehr starke Brille trägt und so eine weitere Vergrößerung stattfindet, bevor das Bild an seinem Auge angelangt.\\
Wir können dieses Phänomen glücklicherweise weiter testen: Die Experimentatorin trägt an diesem Praktikumstag Kontaktlinsen, hat jedoch ihre Brille dabei. Beim Vergleich zwischen Beobachtung mit Kontaktlinsen bzw. Brille bestätigt sich unsere Theorie.\\
Wir verwenden daher die Beobachtungen, die von der Kontaktlinsen-tragenden Experimentatorin durchgeführt werden können.\\
Wir schreiben eine Unsicherheit von 0.5 Einheiten, da die Experimentatorin keine der Distanz wahrhaft genügenden Kontaktlinsen trägt und daher mit dem Auge, das den Maßstab direkt anvisiert nicht scharf genug sehen kann. Wir wissen nicht, ob andere Gruppen (mitsamt Unsicherheit) präzisere Messergebnisse aufgrund ihrer Sehkraft erreichen können.
\end{document}
