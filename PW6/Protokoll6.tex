\documentclass{article}
\usepackage{amsmath}
\usepackage[utf8]{inputenc}
\usepackage{float}
\usepackage{epsfig,graphicx}
\usepackage{xcolor,import}
\usepackage[german]{babel}
\usepackage{textcomp}
\usepackage{mathtools}

\begin{document}
\thispagestyle{empty}
			\begin{center}
			\Large{Fakultät für Physik}\\
			\end{center}
\begin{verbatim}


\end{verbatim}
							%Eintrag des Wintersemesters
			\begin{center}
			\textbf{\LARGE WINTERSEMESTER 2014/15}
			\end{center}
\begin{verbatim}


\end{verbatim}
			\begin{center}
			\textbf{\LARGE{Physikalisches Praktikum 1}}
			\end{center}
\begin{verbatim}




\end{verbatim}

			\begin{center}
			\textbf{\LARGE{PROTOKOLL}}
			\end{center}
			
\begin{verbatim}





\end{verbatim}

			\begin{flushleft}
			\Large{\textbf{Experiment Nr. 6:} Geometrische Optik}\\
							%Experiment Nr. und Titel statt den Punkten eintragen
			\LARGE{}	
			\end{flushleft}

\begin{verbatim}

\end{verbatim}	
							%Eintragen des Abgabedatums, oder des Erstelldatums des Protokolls
			\begin{flushleft}
			\textbf{\Large{Datum:}} \Large{.2014}
			\end{flushleft}
			
\begin{verbatim}
\end{verbatim}
							%Namen der Protokollschreiber
		\begin{flushleft}
			\textbf{\Large{Namen:}} \Large{Veronika Bachleitner, Erik Grafendorfer}
			\end{flushleft}

\begin{verbatim}


\end{verbatim}
							%Kurstag und Gruppennummer, zb. Fr/5
			\begin{flushleft}
			\textbf{\Large{Kurstag/Gruppe:}} \Large{Fr/1}
			\end{flushleft}

\begin{verbatim}






\end{verbatim}
							%Name des Betreuers, das Praktikum betreute.
			\begin{flushleft}
			\LARGE{\textbf{Betreuer:}}	\Large{}	
			\end{flushleft}
\newpage	

\section{Brennweite von Linsen}

\subsection{Aufgabenstellung}
\subsection{Grundlagen}
\subsection{Versuchsaufbau und Methoden}
\subsection{Durchführung}
\subsection{Ergebnisse}
\subsubsection*{Brennweite einer Konvexlinse}
\paragraph{Bei einer Gegenstandsweite}

Stellen g ein\\
Messen b\\
Berechnen f\\
Fehler von f\\
\paragraph{Besselverfahren}
5 Gegenstandsweiten; Messen d und e
\subsubsection*{Brennweite einer Konkavlinse}
Mit Konvexlinse g1 und b1 festsetzen. Abstand d der beiden Linsen. \\
Abstand vergrößern, zweite Bildweite b2 messen und Gegenstandsweiter g2 durch g2=-(b1-d) bestimmen.\\
in linsengleichung einsetzen
MILLIMETERPAPIER STRAHLENGANG AAAAH
\subsubsection*{Brechkraft der Linsen}
\subsection{Diskussion}

\section{Linsenfehler}

\subsection{Aufgabenstellung}
\subsection{Grundlagen}
\subsection{Versuchsaufbau und Methoden}
\subsection{Durchführung}
\subsection{Ergebnisse}
\subsection{Diskussion}

\section{Mikroskop}

\subsection{Aufgabenstellung}
\subsection{Grundlagen}
\subsection{Versuchsaufbau und Methoden}
\subsection{Durchführung}
\subsection{Ergebnisse}
möglichst EINFACHE kombination\\
gesamtvergrößerung messen\\
theoretische: $$V_M=\frac{t}{V_Ob}\frac{s_0}{V_Ok}$$
\subsection{Diskussion}

\section{Fernrohr}

\subsection{Aufgabenstellung}
\subsection{Grundlagen}
\subsection{Versuchsaufbau und Methoden}
\subsection{Durchführung}
\subsection{Ergebnisse}
\subsection{Diskussion}
\end{document}
