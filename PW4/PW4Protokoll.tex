\documentclass{article}
\usepackage{amsmath}
\usepackage[utf8]{inputenc}
\usepackage{float}
\usepackage{epsfig,graphicx}
\usepackage{xcolor,import}
\usepackage[german]{babel}
\usepackage{textcomp}
\usepackage{mathtools}

\begin{document}
\thispagestyle{empty}
			\begin{center}
			\Large{Fakultät für Physik}\\
			\end{center}
\begin{verbatim}


\end{verbatim}
							%Eintrag des Wintersemesters
			\begin{center}
			\textbf{\LARGE WINTERSEMESTER 2014/15}
			\end{center}
\begin{verbatim}


\end{verbatim}
			\begin{center}
			\textbf{\LARGE{Physikalisches Praktikum 1}}
			\end{center}
\begin{verbatim}




\end{verbatim}

			\begin{center}
			\textbf{\LARGE{PROTOKOLL}}
			\end{center}
			
\begin{verbatim}





\end{verbatim}

			\begin{flushleft}
			\textbf{\Large{Experiment (Nr., Titel):}}\\
							%Experiment Nr. und Titel statt den Punkten eintragen
			\LARGE{4. Oberflächenspannung, Viskosität, Hygrometrie, Schmelzwärme}	
			\end{flushleft}

\begin{verbatim}

\end{verbatim}	
							%Eintragen des Abgabedatums, oder des Erstelldatums des Protokolls
			\begin{flushleft}
			\textbf{\Large{Datum:07.11}} \Large{.2014}
			\end{flushleft}
			
\begin{verbatim}
\end{verbatim}
							%Namen der Protokollschreiber
		\begin{flushleft}
			\textbf{\Large{Namen:}} \Large{Veronika Bachleitner, Erik Grafendorfer}
			\end{flushleft}

\begin{verbatim}


\end{verbatim}
							%Kurstag und Gruppennummer, zb. Fr/5
			\begin{flushleft}
			\textbf{\Large{Kurstag/Gruppe:}} \Large{Fr/1}
			\end{flushleft}

\begin{verbatim}






\end{verbatim}
							%Name des Betreuers, das Praktikum betreute.
			\begin{flushleft}
			\LARGE{\textbf{Betreuer:}}	\Large{}	
			\end{flushleft}
\newpage
\section{Oberflächenspannung}
\subsection{Aufgabenstellung}
Wir wollen mithilfe einer Torsionswaage die Oberflächenspannung einer Probeflüßigkeit bestimmen. Mehr nicht. Wir sind ziemlich bescheiden.
\subsection{Grundlagen}
Die Oberflächenspannung resultiert aus der Anziehungsenergie zwischen den Molekülen einer Flüßigkeit. Sie ist proportional zur Oberfläche.

\begin{equation} 
E=\sigma \cdot A
\end{equation}

wobei $\sigma$ die spezifische Oberflächenenergie darstellt.
\subsection{Versuchsaufbau und Methoden}
Wir haben einen Behälter mit Flüßigkeit, eine Torsionswaage, an der ein Metallring zum Untertauchen befestigt ist, und ausreichend Putzutensilien. Wir putzen! Dann, wenn alles sauber ist, messen wir. Und zwar die Kraft K, die auf der zuvor geeichten Torsionswaage angezeigt wird, kurz bevor die sich im untergetauchten Metallring mit Hilfe der Oberflächenspannung festklammernde Flüßigkeitsschicht reisst. Daraus berechnen wir uns $\sigma$:
\begin{equation}
\sigma=\frac{K}{\pi(d_1+d_2)}
\end{equation}
\subsection{Durchführung}
Wir eichen erst die Torsionswaage von 0.1g bis 1.4g in 0.1g Schritten. Dabei notieren wir die angezeigte Skalenauslenkung bei den jeweiligen Kräften und legen dann eine Regressionsgrade durch diese Punkte um die Skalenanteile in eine Kraft umrechnen zu können.
Bei der Durchführung ergaben sich keine Probleme.
\subsection{Ergebnisse}

\subsection{Diskussion}
\section{Dynamische Viskosität}
\subsection{Aufgabenstellung}
\subsection{Grundlagen}
\subsection{Versuchsaufbau und Methoden}
\subsection{Durchführung}
\subsection{Ergebnisse}
\subsection{Diskussion}
\section{Luftfeuchtigkeit}
\subsection{Aufgabenstellung}
\subsection{Grundlagen}
\subsection{Versuchsaufbau und Methoden}
\subsection{Durchführung}
\subsection{Ergebnisse}
\subsection{Diskussion}
\section{Schmelzwärme von Eis}
\subsection{Aufgabenstellung}
\subsection{Grundlagen}
\subsection{Versuchsaufbau und Methoden}
\subsection{Durchführung}
\subsection{Ergebnisse}
\subsection{Diskussion}
\end{document}
