\documentclass{article}
\usepackage{amsmath}
\usepackage[utf8]{inputenc}
\usepackage{float}
\usepackage{epsfig,graphicx}
\usepackage{xcolor,import}
\usepackage[german]{babel}
\usepackage{textcomp}
\usepackage{mathtools}

\begin{document}
\thispagestyle{empty}
			\begin{center}
			\Large{Fakultät für Physik}\\
			\end{center}
\begin{verbatim}


\end{verbatim}
							%Eintrag des Wintersemesters
			\begin{center}
			\textbf{\LARGE WINTERSEMESTER 2014/15}
			\end{center}
\begin{verbatim}


\end{verbatim}
			\begin{center}
			\textbf{\LARGE{Physikalisches Praktikum 1}}
			\end{center}
\begin{verbatim}




\end{verbatim}

			\begin{center}
			\textbf{\LARGE{PROTOKOLL}}
			\end{center}
			
\begin{verbatim}





\end{verbatim}

			\begin{flushleft}
			\textbf{\Large{Experiment (Nr., Titel):}}\\
							%Experiment Nr. und Titel statt den Punkten eintragen
			\LARGE{4. Oberflächenspannung, Viskosität, Hygrometrie, Schmelzwärme}	
			\end{flushleft}

\begin{verbatim}

\end{verbatim}	
							%Eintragen des Abgabedatums, oder des Erstelldatums des Protokolls
			\begin{flushleft}
			\textbf{\Large{Datum:07.11}} \Large{.2014}
			\end{flushleft}
			
\begin{verbatim}
\end{verbatim}
							%Namen der Protokollschreiber
		\begin{flushleft}
			\textbf{\Large{Namen:}} \Large{Veronika Bachleitner, Erik Grafendorfer}
			\end{flushleft}

\begin{verbatim}


\end{verbatim}
							%Kurstag und Gruppennummer, zb. Fr/5
			\begin{flushleft}
			\textbf{\Large{Kurstag/Gruppe:}} \Large{Fr/1}
			\end{flushleft}

\begin{verbatim}






\end{verbatim}
							%Name des Betreuers, das Praktikum betreute.
			\begin{flushleft}
			\LARGE{\textbf{Betreuer:}}	\Large{}	
			\end{flushleft}
\newpage
\section{Oberflächenspannung}
\subsection{Aufgabenstellung}
Wir wollen mithilfe einer Torsionswaage die Oberflächenspannung einer Probeflüßigkeit bestimmen. Mehr nicht. Wir sind ziemlich bescheiden.
\subsection{Grundlagen}
Die Oberflächenspannung resultiert aus der Anziehungsenergie zwischen den Molekülen einer Flüßigkeit. Sie ist proportional zur Oberfläche.

\begin{equation} 
E=\sigma \cdot A
\end{equation}

wobei $\sigma$ die spezifische Oberflächenenergie darstellt.
\subsection{Versuchsaufbau und Methoden}
Wir haben einen Behälter mit Flüßigkeit, eine Torsionswaage, an der ein Metallring zum Untertauchen befestigt ist, und ausreichend Putzutensilien. Wir putzen! Dann, wenn alles sauber ist, messen wir. Und zwar die Kraft K, die auf der zuvor geeichten Torsionswaage angezeigt wird, kurz bevor die sich im untergetauchten Metallring mit Hilfe der Oberflächenspannung festklammernde Flüßigkeitsschicht reisst. Daraus berechnen wir uns $\sigma$:
\begin{equation}
\sigma=\frac{K}{\pi(d_1+d_2)}
\end{equation}
\subsection{Durchführung}
Wir eichen erst die Torsionswaage von 0.1g bis 1.4g in 0.1g Schritten. Dabei notieren wir die angezeigte Skalenauslenkung bei den jeweiligen Kräften und legen dann eine Regressionsgrade durch diese Punkte um die Skalenanteile in eine Kraft umrechnen zu können.
Bei der Durchführung ergaben sich keine Probleme.
\subsection{Ergebnisse}

\subsection{Diskussion}
\section{Dynamische Viskosität}
\subsection{Aufgabenstellung}
In diesem Experiment bestimmen wir die dynamische Viskosität $\eta$ einer Testflüssigkeit auf zwei verschiedene Arten: Einmal mit der Stokes'schen Kugelfallmethode und einmal mit dem Höppler-Viskosimeter.
\subsection{Grundlagen}
\textbf{Viskosität}\\
Die Viskosität ist die Zähigkeit einer Flüssigkeit. Sie entsteht durch die Reibungskräfte in Flüssigkeiten, die daraus folgen, dass die Moleküle nicht fest geordnet sind und sich gegeneinander verschieben lassen.\\
Ist ein Fluid dickflüssig, ist die Viskosität größer; ist es dünnflüssig, ist die Viskosität niedriger.\\
\begin{gather}
F=\eta A \frac{dv}{dx}\\
[\eta]=N s/m^2 = kg m^{-1}s^{-1} = Pa s
\end{gather}

\subsection{Versuchsaufbau und Methoden}
\textbf{Stokes'sche Kugelfallmethode:}\\
Die Reibungskraft wird hier geschrieben als:
$F = -6 \pi \eta v r$, wobei r der Kugelradius ist. \\
Aus dem Anleitungstext stehen uns noch folgende zusätzliche Daten zur Verfügung:\\
\\
Gefäßradius: $R=15mm$\\ 
Dichte der Flüssigkeit: $\rho=(1225 \pm 5) kg/m^3$\\
Fallstrecke: $a=(150.0 \pm 0.5)mm$\\
Kugelradius 1: $r_1=(150.0 \pm 0.5)mm$\\
Kugelmasse 1: $m_1=(6.6 \pm 0.1)mg$\\
Kugelradius 2: $r_2=(0.794 \pm 0.020)mm$\\
Kugelmasse 2: $m_2=(16.1 \pm 0.1)mg$\\
\\
\textbf{Höppler-Viskosimeter:}\\
Durchmesser Kugel 3: $2r_3=15.604mm$\\
Kugelmasse 3: $m_3=16.1760g$\\
Kugeldichte 3: $\rho=8.131 g/cm^3$\\
Konstante K: $K=0.09208mPa cm^3 g^{-1}$\\

\subsection{Ergebnisse}
\textbf{Stokes'sche Kugelfallmethode:}\\
\textbf{Fallzeit:}//
$$t_1=$$//
$$t_2=$$//
//
Temperatur: $T=$//

\textbf{Höppler-Viskosimeter:}\\
\subsection{Diskussion}
\section{Luftfeuchtigkeit}
\subsection{Aufgabenstellung}
Wir messen die Luftfeuchtigkeit im Labor mit einem Aspirationspsychrometer nach Aßmann.
\subsection{Grundlagen}
\textit{Luftfeuchtigkeit} bezeichnet den Anteil von Wasserdampf am betrachteten Gasgemisch. \\
\textit{Partialdruck} nennt man den Druck einer einzigen Komponente eines Gasgemisches, wenn sonst keine andere Komponente anwesend wäre. Der Partialdruck eines Gases kann einen von der Temperatur abhängigen Maximaldruck nicht übersteigen, diesen nennt man den \textit{Sättigungsdampfdruck.}	\\
\textit{Verdunstungswärme} bezeichnet die Energie, die der Luft entzogen wird, wenn eine Flüßigkeit verdunstet. \\
\subsubsection*{Die Psychrometerformel}
Wir verwenden diese Formel um aus der gemessenen Differenz zwischen der Temperatur des trockenen Thermometers $T_t$ und der des feuchten $T_f$ den Partialdruck des Wasserdampfes in der Luft zu bestimmen.
\begin{equation}
\label{psychrometerformel}
p_w(T_t)=p_{w,max}(T_f)-\frac{p_L\cdot c_p \cdot M_L}{q_v\cdot M_w} \cdot (T_t - T_f)
\end{equation}
Bei einem mittleren Luftdruck auf Meereshöhe von $p_L$= 101.325kPa, einer spezifischen Verdampfungswärme von Wasser $q_v$ = 2440kJ/kg bei 25°C, einer isobaren spezifischen Wärmekapazität von Luft $c_p=1.005kJkg^{-1}K^{-1}$ und einer molaren Masse von Luft $M_L$=28.9644g/mol entspricht $\frac{p_L\cdot c_p \cdot M_L}{q_v\cdot M_w} \approx $ 67Pa/K. [laut Aufgabenstellung.]
\subsection{Versuchsaufbau und Methoden}
Wir verwenden ein Aspirationspsychrometer nach Aßmann. Dabei wird eines der beiden Thermometer befeuchtet, eines bleibt trocken; es wird für eine konstante Ventilation gesorgt. Am feuchten Thermometer verdunstet Wasser, und zwar je mehr, umso trockener die Luft ist. Dabei wird dem Thermometer \textit{Verdungstungswärme} entzogen; die Temperatur sinkt, und zwar um so mehr, je trockener die Luft ist! Aus der Temperaturdifferenz zum trockenen Thermometer können wir die absolute Luftfeuchtigkeit bestimmen. 
\subsection{Durchführung}
\subsection{Ergebnisse}
\subsection{Diskussion}
\section{Schmelzwärme von Eis}
\subsection{Aufgabenstellung}
Wir bestimmen die Schmelzwärme von Eis mittels Mischungsmethode mit dem Kalorimeter.
\subsection{Grundlagen}
Latente Wärmemengen werden bei Änderung des Aggregatszusandes frei, wenn das Volumen, aber nicht die Temperatur geändert wird. \\
\textbf{Mischungsmethode:}
Es werden zwei verschiedenen Massen $m_i$ (eine flüssig, eine flüssig oder fest) mit unterschiedlicher Temperatur $T_i$ und spezifischer Wärme $c_i$ gemischt. Sobald sich eine Mischungstemperatur $T_m$ eingestellt hat, lässt sich für die von $m_1$ und dem Kalorimeter abgegebene Wärmemenge:\\
$$\Delta Q_1=(C_k + c_1m_1)(T_1 - T_m)$$\\
und für die von $m_2$ aufgenommene Wärmemenge: \\
$$\Delta Q_2=c_2m_2(T_m - T_2)$$\\
Aus dem Energieerhaltungssatz: \\
$$\Delta Q_1=\Delta Q_2=\Delta Q$$\\
\subsection{Versuchsaufbau und Methoden}

\subsection{Ergebnisse}
\subsection{Diskussion}
\end{document}
