\documentclass{article}
\usepackage{amsmath}
\usepackage[utf8]{inputenc}
\usepackage{float}
\usepackage{epsfig,graphicx}
\usepackage{xcolor,import}
\usepackage[german]{babel}
\usepackage{textcomp}
\usepackage{mathtools}

\begin{document}
\thispagestyle{empty}
			\begin{center}
			\Large{Fakultät für Physik}\\
			\end{center}
\begin{verbatim}


\end{verbatim}
							%Eintrag des Wintersemesters
			\begin{center}
			\textbf{\LARGE WINTERSEMESTER 2014/15}
			\end{center}
\begin{verbatim}


\end{verbatim}
			\begin{center}
			\textbf{\LARGE{Physikalisches Praktikum 1}}
			\end{center}
\begin{verbatim}




\end{verbatim}

			\begin{center}
			\textbf{\LARGE{PROTOKOLL}}
			\end{center}
			
\begin{verbatim}





\end{verbatim}

			\begin{flushleft}
			\textbf{\Large{Experiment Nr.7:} Brechung, Dispersion, Refraktometrie}\\
							%Experiment Nr. und Titel statt den Punkten eintragen
			\LARGE{}	
			\end{flushleft}

\begin{verbatim}

\end{verbatim}	
							%Eintragen des Abgabedatums, oder des Erstelldatums des Protokolls
			\begin{flushleft}
			\textbf{\Large{Datum:}} \Large{28.11.2014}
			\end{flushleft}
			
\begin{verbatim}
\end{verbatim}
							%Namen der Protokollschreiber
		\begin{flushleft}
			\textbf{\Large{Namen:}} \Large{Veronika Bachleitner, Erik Grafendorfer}
			\end{flushleft}

\begin{verbatim}


\end{verbatim}
							%Kurstag und Gruppennummer, zb. Fr/5
			\begin{flushleft}
			\textbf{\Large{Kurstag/Gruppe:}} \Large{Fr/1}
			\end{flushleft}

\begin{verbatim}






\end{verbatim}
							%Name des Betreuers, das Praktikum betreute.
			\begin{flushleft}
			\LARGE{\textbf{Betreuer:}}	\Large{WIECZOREK}	
			\end{flushleft}
\newpage	

\section{Allgemeine Grundlagen}
Laser: Light Amplification by Stimulated Emission of Radiation

\section{Beugung am Spalt und Doppelspalt}

\subsection{Aufgabenstellung}
Wir vermessen das Beugungsbild hinter einem mit monochromatischem Licht  beleuchteten Einzelspalt und Doppelspalt. \\
Wir berechnen uns daraus jeweils die Spaltbreite und den Spaltabstand.
\subsection{Grundlagen}
\subsection{Versuchsaufbau und Methoden}
Wir verwenden einen He-Ne-Laser (Wellenlänge = 632.8nm, Strahldivergenz = 1.2 mrad)
\subsection{Durchführung}
\subsection{Ergebnisse}
Wir verwenden einen Laser mit Wellenlänge $\lambda=632.8nm$.\\
\\
\textbf{Einzelspalt:}\\
\\
$\alpha_n=\frac{d}{D}$, wobei d der Abstand der Minima vom Nullpunkt und D der Abstand des Spalts vom Schirm. Hier ist $D=1019mm$\\

\begin{tabular}{|r|l|l|}
\hline
n & d & $\alpha_n$\\
\hline
1 & 5.70mm & 0.0056\\
2 & 11.35mm & 0.0111\\
3 & 17.55mm & 0.0172\\
4 & 22.90mm & 0.0225\\
5 & 28.30mm & 0.0278\\
\hline
\end{tabular}
\vspace{0.8cm}
\\Um die Spaltbreite a zu erhalten verwenden wir folgende Beziehung:
$$n\lambda=a sin(\alpha_n)$$\\
Wir lassen uns nun a mithilfe einer linearen Regression in QTI-Plot berechnen:\\
Dabei wird die Formel $Y=AX+B$ verwendet, wobei bei uns $Y=n\lambda$, $X=sin(\alpha_n)\approx\frac{d}{D}$ und A die gesuchte Spaltbreite.\\

\begin{quote}
From x = 5.5937193326790e-03 to x = 2.7772325809620e-02\\
B (y-intercept) = -1.3683403922226e+01 +/- 3.2623800650797e+01\\
A (slope) = 1.1354388045435e+05 +/- 1.7547210828639e+03\\
\end{quote}

Daraus die Spaltbreite: 
$$\boxed{a=0.1135 \pm 0.0018mm}$$\\

\textbf{Doppelspalt:}\\
\\
Distanz der Minima vom Nullpunkt:\\
$\alpha_n=\frac{d}{D}$, wobei d der Abstand der Minima vom Nullpunkt und D der Abstand des Spalts vom Schirm. Hier ist $D=1000mm$\\
\begin{tabular}{|r|l|l|}
\hline
n & d & $\alpha_n$\\
\hline
1 & 5.30mm & 0.0053\\
2 & 10.90mm & 0.0109\\
3 & 16.15mm & 0.0162\\
4 & 21.10mm & 0.0211\\
\hline
\end{tabular}
\vspace{0.8cm}
\\Lasse die Spaltbreite a wieder von QTI-Plot berechnen:\\
From x = 5.3000000000000e-03 to x = 2.1100000000000e-02\\
B (y-intercept) = -2.2813544750395e+01 +/- 3.4253583389353e+01\\
A (slope) = 1.2009830082323e+05 +/- 2.3457335286015e+03\\
\\
Also ist die Spaltbreite
$$\boxed{a=0.1201 \pm 0.0024 mm}$$
\\
Berechnung von b: Setze folgende Formeln gleich:\\
$$sin(\alpha_{min,k})=\frac{\lambda}{2b}(2k+1)$$
$$sin(\alpha_{min,n})=\frac{n\lambda}{a}$$
Daraus erhalten wir für b:
$$b=\frac{\lambda a}{2n\lambda}(2k+1)=\frac{a(2k+1)}{2n}=\frac{9a}{2}$$
nun setzen wir k=4, da wir im Hauptmaximum 8 Subminima gezählt haben:
$$\boxed{b=0.540 \pm 0.011 mm}$$
\section{Wellenlängenmessung mit dem Gitter}

\subsection{Aufgabenstellung}
Wir vermessen das Beugungsbild eines durch eine Spektrallampe beleuchteten Beugungsgitters. Wir bestimmen daraus Wellenlänge für die drei Spektrallinien und vergleichen unser Ergebnis mit den Literaturwerten.
\subsection{Grundlagen}
\subsection{Versuchsaufbau und Methoden}
\subsection{Durchführung}
\subsection{Ergebnisse}
Die Beziehung zwischen Wellenlänge, Gitterkonstante, Ordnung und Beugungswinkel:\\
$$\lambda=\frac{sin(\vartheta)d}{n}$$
wobei die Gitterkonstante $d=10^{-5}$.\\
\\
\begin{center}
\begin{tabular}{|r|r|l|l|}
\hline
& Ordnung & Winkel & $\lambda$\\
\hline
blau & & &\\
& 1 & 2°30' & 436nm\\
& 2 & 4°58' & 433nm\\
& 3 & 7°32' & 436nm\\
\hline
grün & & &\\
& 1 & 3°4' & 535nm\\
& 2 & 6°16' & 546nm\\
& 3 & 9°23' & 543nm\\
\hline
gelb & & &\\
& 1 & 3°18' & 576nm\\
& 2 & 6°39' & 579nm\\
& 3 & 10°0' & 579nm\\
\hline
\end{tabular}
\end{center}
\vspace{0.8cm}
Daraus die gemittelten Wellenlängen:\\
blau: 435nm\\
grün: 541nm\\
gelb: 578nm\\

\subsection{Diskussion}
Es handelt sich um ein Hg-Emissionsspektrum.\\
\\
Literaturwerte für Hg:\\
435.84nm bzw. 434.75nm bzw. 433.92 für blau\\
546.07nm für grün\\
576.96nm bzw. 579.07nm für gelb\\

\section{Newtonsche Ringe}

\subsection{Aufgabenstellung}
\subsection{Grundlagen}
\subsection{Versuchsaufbau und Methoden}
\subsection{Durchführung}
\subsection{Ergebnisse}
Wellenlänge beim grünen Filter: $\lambda=518 \pm 3nm$
$$R=\frac{r_k^2}{k\lambda}$$

Die Newtonschen Ringe der Reflexion sind auf dem Schirm 2x vergrößert. Bei der Transmission sind sie 4x vergrößert.

\begin{tabular}{r|c|l}
k & $r_k$ & \\
1 & &\\
2 & &\\
3 & &\\
4 & &\\
5 & &\\
6 & &\\
7 & &\\

\end{tabular}

\end{document}
