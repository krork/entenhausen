\documentclass{article}
\usepackage{amsmath}
\usepackage[utf8]{inputenc}
\usepackage{float}
\usepackage{epsfig,graphicx}
\usepackage{xcolor,import}
\usepackage[german]{babel}
\usepackage{textcomp}
\usepackage{mathtools}

\begin{document}
\thispagestyle{empty}
			\begin{center}
			\Large{Fakultät für Physik}\\
			\end{center}
\begin{verbatim}


\end{verbatim}
							%Eintrag des Wintersemesters
			\begin{center}
			\textbf{\LARGE WINTERSEMESTER 2014/15}
			\end{center}
\begin{verbatim}


\end{verbatim}
			\begin{center}
			\textbf{\LARGE{Physikalisches Praktikum 1}}
			\end{center}
\begin{verbatim}




\end{verbatim}

			\begin{center}
			\textbf{\LARGE{PROTOKOLL}}
			\end{center}
			
\begin{verbatim}





\end{verbatim}

			\begin{flushleft}
			\textbf{\Large{Experiment Nr.7:} Brechung, Dispersion, Refraktometrie}\\
							%Experiment Nr. und Titel statt den Punkten eintragen
			\LARGE{}	
			\end{flushleft}

\begin{verbatim}

\end{verbatim}	
							%Eintragen des Abgabedatums, oder des Erstelldatums des Protokolls
			\begin{flushleft}
			\textbf{\Large{Datum:}} \Large{28.11.2014}
			\end{flushleft}
			
\begin{verbatim}
\end{verbatim}
							%Namen der Protokollschreiber
		\begin{flushleft}
			\textbf{\Large{Namen:}} \Large{Veronika Bachleitner, Erik Grafendorfer}
			\end{flushleft}

\begin{verbatim}


\end{verbatim}
							%Kurstag und Gruppennummer, zb. Fr/5
			\begin{flushleft}
			\textbf{\Large{Kurstag/Gruppe:}} \Large{Fr/1}
			\end{flushleft}

\begin{verbatim}






\end{verbatim}
							%Name des Betreuers, das Praktikum betreute.
			\begin{flushleft}
			\LARGE{\textbf{Betreuer:}}	\Large{WIECZOREK}	
			\end{flushleft}
\newpage	

\section{Allgemeine Grundlagen}
Laser: Light Amplification by Stimulated Emission of Radiation

\section{Beugung am Spalt und Doppelspalt}

\subsection{Aufgabenstellung}
Wir vermessen das Beugungsbild hinter einem mit monochromatischem Licht  beleuchteten Einzelspalt und Doppelspalt. \\
Wir berechnen uns daraus jeweils die Spaltbreite und den Spaltabstand.
\subsection{Grundlagen}
\subsection{Versuchsaufbau und Methoden}
Wir verwenden einen He-Ne-Laser (Wellenlänge = 632.8nm, Strahldivergenz = 1.2 mrad)
\subsection{Durchführung}
\subsection{Ergebnisse}
Ordnung n, Wellenlänge $\lambda$, Spaltbreite a, Beugungswinkel $\alpha_n$ der n-ten Ordnung\\
Diagramm mit Ordinate $n\lambda$ und Abszisse $\alpha_n$ für min. 6 Ordnungen. -> Lineare Regression, daraus erhalten wir die Spaltbreite a. (Einzelspalt)

$$n\lambda=a sin(\alpha_n)$$\\
Daraus die Spaltbreite a:
$$a=\frac{n\lambda}{sin(\alpha_n)}$$

$$sin(\alpha_{min,k})=\frac{\lambda}{2b}(2k+1)$$
$$\delta=2\pi\frac{\Delta x}{\lambda}=2\pi\frac{b sin(\alpha)}{\lambda}$$\\
daher Spaltbreite b beim Doppelspalt:\\
$$b=\frac{\lambda (2k+1)}{2sin(\alpha_{min,k})}$$
oder
$$b=\delta\lambda(2\pi sin(\alpha))^{-1}$$
\section{Wellenlängenmessung mit dem Gitter}

\subsection{Aufgabenstellung}
Wir vermessen das Beugungsbild eines durch eine Spektrallampe beleuchteten Beugungsgitters. Wir bestimmen daraus Wellenlänge für die drei Spektrallinien und vergleichen unser Ergebnis mit den Literaturwerten.
\subsection{Grundlagen}
\subsection{Versuchsaufbau und Methoden}
\subsection{Durchführung}
\subsection{Ergebnisse}
1.,2.,3. Ordnung für 3 Spektrallinien vermessen.\\
Beziehung zwischen Gitterkonstante, Ordnung & Beugungswinkel.\\
Literaturwerte\\
Um welches Emissionsspektrum handelt es sich?\\

\section{Newtonsche Ringe}

\subsection{Aufgabenstellung}
\subsection{Grundlagen}
\subsection{Versuchsaufbau und Methoden}
\subsection{Durchführung}
\subsection{Ergebnisse}

\end{document}
