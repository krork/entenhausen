\documentclass{article}
\usepackage{amsmath}
\usepackage[utf8]{inputenc}
\usepackage{float}
\usepackage{epsfig,graphicx}
\usepackage{xcolor,import}
\usepackage[german]{babel}
\usepackage{textcomp}
\usepackage{mathtools}

\begin{document}
\thispagestyle{empty}
			\begin{center}
			\Large{Fakultät für Physik}\\
			\end{center}
\begin{verbatim}


\end{verbatim}
							%Eintrag des Wintersemesters
			\begin{center}
			\textbf{\LARGE WINTERSEMESTER 2014/15}
			\end{center}
\begin{verbatim}


\end{verbatim}
			\begin{center}
			\textbf{\LARGE{Physikalisches Praktikum 1}}
			\end{center}
\begin{verbatim}




\end{verbatim}

			\begin{center}
			\textbf{\LARGE{PROTOKOLL}}
			\end{center}
			
\begin{verbatim}





\end{verbatim}

			\begin{flushleft}
			\textbf{\Large{Experiment (Nr.7 , Titel):} Brechung, Dispersion, Refraktometrie}\\
							%Experiment Nr. und Titel statt den Punkten eintragen
			\LARGE{}	
			\end{flushleft}

\begin{verbatim}

\end{verbatim}	
							%Eintragen des Abgabedatums, oder des Erstelldatums des Protokolls
			\begin{flushleft}
			\textbf{\Large{Datum:}} \Large{28.11.2014}
			\end{flushleft}
			
\begin{verbatim}
\end{verbatim}
							%Namen der Protokollschreiber
		\begin{flushleft}
			\textbf{\Large{Namen:}} \Large{Veronika Bachleitner, Erik Grafendorfer}
			\end{flushleft}

\begin{verbatim}


\end{verbatim}
							%Kurstag und Gruppennummer, zb. Fr/5
			\begin{flushleft}
			\textbf{\Large{Kurstag/Gruppe:}} \Large{Fr/1}
			\end{flushleft}

\begin{verbatim}






\end{verbatim}
							%Name des Betreuers, das Praktikum betreute.
			\begin{flushleft}
			\LARGE{\textbf{Betreuer:}}	\Large{}	
			\end{flushleft}
\newpage	

\section{Dispersionskurve eines Glasprismas}

\subsection{Aufgabenstellung}
Wir bestimmen mit einem Prismenspektrometer die Winkel der minimalen Ablenkung für 5 verschiedene Spektrallinien einer Metalldampflampe.\\ Dann bestimmen wir die Wellenlängen dieser Spektrallinien aus ihren Emissionsmaxima mit einem automatischen Gitterspektrometer.\\ Schließlich zeichnen wir die Dispersionskurve, also die Brechzahlen über die Wellenlängen, des verwendeten Glasprismas. 
\subsection{Grundlagen}
\subsubsection*{Das Brechungsgesetz}
\begin{equation}
\label{brechungsgesetz}
\frac{sin\alpha_1}{sin\alpha_2}=\frac{c_1}{c_2}=n_{21}=\frac{n2}{n1}=\frac{1}{n_{12}}
\end{equation}
$\alpha$ bezeichnen die Brechungswinkel, $c_i$ die Lichtgeschwindigkeiten in den Medien, $n_i$ die Brechzahlen in den Medien, und $n_{ij}$ die relativen Brechzahlen. \\
\\
\subsubsection*{Den Brechungsindex bestimmen}
Aus dem Winkel der minimalen Ablenkung $\delta_{min}$ und $\varepsilon$, dem brechenden Winkel des Prismas, können wir die Brechzahlen \textit{n} für die verschiedenen Wellenlängen bestimmen:
\begin{equation}
\label{brechzahl}
n=\frac{sin\frac{\delta_{min}+\varepsilon}{2}}{sin\frac{\varepsilon}{2}}
\end{equation}
\subsection{Versuchsaufbau und Methoden}
Wir verwenden einen drehbaren Prismentisch. Auf diesen leuchten wir mit einer Metalldampflampfe durch eine Kollimatorlinse. Erst messen wir den Winkel $\varphi_1$, in dem das Licht ohne Ablenkung durch ein Prisma durchtritt, dann messen wir $\varphi_2$, nachdem das Licht durch ein Prisma getreten ist. Wir messen durch ein fix mit einem Winkelnonius verbundenes Fernrohr, in das die vom Prisma gebrochenen Strahlen fallen. Wir drehen dabei den Tisch so lange, bis sich die Spektrallinien im Okular des Fernrohres nicht mehr bewegen und schließlich in die andere Richtung wandern - am Umkehrpunkte können wir $\varphi_2$ messen, dessen Differenz zu $\varphi_1$ den Winkel der minimalen Ablenkung $\delta_{min}$ für die betrachtete Wellenlänge bildet. Daraus können wir dann für jede der Spektrallinien die Brechzahl des Prismas mit (\ref{brechzahl}) gewinnen.\\
\\
Dann ermitteln wir die Wellenlängen der Spektrallinien, die wir zuvor beobachtet haben, mit einem automatischen Gitterspektrometer und der Software Spectra Suite um die Dispersionskurve des Prismas zu den ermittelten Brechzahlen zeichnen zu können.
\subsection{Durchführung}
\subsection{Ergebnisse}
$\varphi_1$ ist der Winkel des Lichtstrahls ohne Prisma. \\
$\varphi_1$ - $\varphi_2$ = $\delta_{min}$

Winkel:

Wellenlängen:


\subsection{Diskussion}

\section{Lichtbrechung an einer planparallelen Platte}

\subsection{Aufgabenstellung}
Wir bestimmen die Brechung einer planparellelen Platte indem wir ihre scheinbare Dicke in einem Mikroskop betrachten und mit ihrer wirklichen, mit einer Mikrometerschraube gemessenen, vergleichen.
\subsection{Grundlagen}
Beim Durchgang eines Lichtstrahls durch eine transparente, planparallele Platte wird der Strahl parallel um die Distanz $\delta$ verschoben. Wenn AB die Strecke zwischen Eintritts und Austrittspunkt bedeutet, \textit{d} die tatsächliche Dicke der Platte, der Einfallswinkel $\varepsilon$ und der Brechungswinkel $\varepsilon '$, dann gilt:
\begin{gather*}
sin(\varepsilon-\varepsilon ') = \frac{\delta}{AB} \\
cos \varepsilon ' = \frac{d}{AB}  
\end{gather*}
\begin{equation}
\label{distanzdelta}
\delta =\frac{d\cdot sin(\varepsilon - \varepsilon ')}{cos\varepsilon '}
\end{equation}
Ist weiterhin BC die Distanz entlang dem Lot auf der Platte zwischen dem Austrittspunkt B und dem Schnittpunkt des verlängerten Eintrittsstrahles mit dem Lot der Punkt C, so scheint die Dicke der Platte um diese Distanz verringert. Ihre scheinbare Dicke \textit{b} entspricht der Differenz zwischen ihrer tatsächlichen Dicke \textit{d} und BC. Aus der Geometrie folgt die Brechzahl: 
\begin{equation}
\label{brechzahlplatte}
\frac{tan\varepsilon}{tan\varepsilon '}=\frac{d}{b}\approx\frac{sin\varepsilon}{sin\varepsilon '} = n
\end{equation}
\subsection{Versuchsaufbau und Methoden}
Wir verwenden ein Mikroskop im Durchlichtverfahren. Durch den Feintrieb der Hebung des Objektivtisches können wir den Abstand, um den wir die planparellele Platte heben müssen, damit wir von einem scharfen Bild auf der Oberseite zu einem scharfen Bild auf der Unterseite kommen, messen. Dieser Abstand ist die scheinbare Dicke \textit{b}. Wir messen sie an 5 verschiedenen Stellen und bilden den Mittelwert. Dann bestimmten wir die tatsächliche Dicke \textit{d} der Platte mit einer Mikrometerschraube. Aus (\ref{brechzahlplatte}) bestimmen wir dann die Brechzahl der Platte.
\subsection{Durchführung}
\subsection{Ergebnisse}
Scheinbare Dicke b:

Tatsächliche Dicke d:


\subsection{Diskussion}

\section{Refraktometrie}

\subsection{Aufgabenstellung}
Wir messen den Winkel der Totalreflexion mit einem Abbe-Refraktometer erst für sein Messprisma ohne Probeflüßigkeit, dann mit Probeflüßigkeit. Daraus berechnen wir die Brechungsindizes des Prismas \textit{N} und der Flüßigkeit \textit{n}. Schließlich messen wir den Brechungsindex \textit{n} direkt mit einem anderen Refraktometer. 
\subsection{Grundlagen}
Zuerst der Brechungsindex N des Messprismas. Hierbei verwenden wir den brechenden Winkel dieses Prismas, $\varphi$=61°.
\begin{equation}
\label{NMessprisma}
N=\sqrt{(\frac{cos\varphi + sin i}{sin \varphi})^2+1}
\end{equation}
i ist der Winkel der Totalreflexion ohne Probeflüßigkeit. \\
Daraus bekommen wir mit dem Winkel der Totalreflexion mit Probeflüßigkeit \textit{e}:
\begin{align}
n=N \cdot sin e \\
sin i= N \cdot sin r \\
\varphi = r + e
\end{align}
n ist der Brechungsindex der Probeflüßigkeit, N der des Messprismas und e der Winkel 
\subsection{Versuchsaufbau und Methoden}
Wir verwenden ein Abbe-Refraktometer
\subsection{Durchführung}
\subsection{Ergebnisse}

Winkel i ohne Flüßigkeit:

Winkel e mit Flüßigkeit:

\subsection{Diskussion}
\end{document}
