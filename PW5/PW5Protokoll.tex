\documentclass{article}
\usepackage{amsmath}
\usepackage[utf8]{inputenc}
\usepackage{float}
\usepackage{epsfig,graphicx}
\usepackage{xcolor,import}
\usepackage[german]{babel}
\usepackage{textcomp}
\usepackage{mathtools}

\begin{document}
\thispagestyle{empty}
			\begin{center}
			\Large{Fakultät für Physik}\\
			\end{center}
\begin{verbatim}


\end{verbatim}
							%Eintrag des Wintersemesters
			\begin{center}
			\textbf{\LARGE WINTERSEMESTER 2014/15}
			\end{center}
\begin{verbatim}


\end{verbatim}
			\begin{center}
			\textbf{\LARGE{Physikalisches Praktikum 1}}
			\end{center}
\begin{verbatim}




\end{verbatim}

			\begin{center}
			\textbf{\LARGE{PROTOKOLL}}
			\end{center}
			
\begin{verbatim}





\end{verbatim}

			\begin{flushleft}
			\textbf{\Large{Experiment (Nr., Titel):}}\\
							%Experiment Nr. und Titel statt den Punkten eintragen
			\LARGE{5. Gasthermometer, Adiabatenexponent (Rüchardt), Dampfdichte nach Viktor Meyer}	
			\end{flushleft}

\begin{verbatim}

\end{verbatim}	
							%Eintragen des Abgabedatums, oder des Erstelldatums des Protokolls
			\begin{flushleft}
			\textbf{\Large{Datum: 14.11}}\Large{.2014}
			\end{flushleft}
			
\begin{verbatim}
\end{verbatim}
							%Namen der Protokollschreiber
		\begin{flushleft}
			\textbf{\Large{Namen:}} \Large{Veronika Bachleitner, Erik Grafendorfer}
			\end{flushleft}

\begin{verbatim}


\end{verbatim}
							%Kurstag und Gruppennummer, zb. Fr/5
			\begin{flushleft}
			\textbf{\Large{Kurstag/Gruppe:}} \Large{Fr/1}
			\end{flushleft}

\begin{verbatim}






\end{verbatim}
							%Name des Betreuers, das Praktikum betreute.
			\begin{flushleft}
			\LARGE{\textbf{Betreuer:}}	\Large{SETMAN}	
			\end{flushleft}
\newpage
\section{Allgemeine Grundlagen}
Das Ideale Gas ist ein Modell, bei dem nur Wechselwirkungen durch Stöße der Teilchen untereinander und mit den Wänden angenommen werden.
\textbf{Gleichung des Idealen Gases:}
\begin{equation}
\label{gasgleichung}
pV=nRT
\end{equation}
Experimentell sind Ideale Gase solche, für die in guter Näherung das\\ \textbf{Boyle-Mariotte'sche Gesetz:}
\begin{equation}
\label{boyle-mariotte}
pV=const
\end{equation}
und das \textbf{Gay-Lussac'sche Gesetz:}
\begin{equation}
\label{gay-lussac}
p(t_C)=p(0)(1+\gamma_p t_C)
\end{equation}
erfüllt sind. (Aus \textit{Wagner, Reischl, Steiner: Einführung in die Physik})\\
\\
Hier ist $n$ die Anzahl der Mole der vorliegenden Substanzmenge und weiters
\begin{flushleft}
\begin{tabular}{|l|l|}
\hline Druck & $[p]=Pa$\\
\hline Volumen & $[V]=m^3$\\
\hline Gaskonstante & $R=8.3143 J K^{-1} mol^{-1}$, $R=k_B N_A$\\
\hline Boltzmannkonstante & $k_B=1.3806488*10^{23}J/K$\\
\hline Avogadro'sche oder &\\
Loschmidt'sche Zahl & $N_A=6.02214179*10^{23}$\\
\hline Absolute Temperatur & $[T]=^\circ C$\\
\hline
\end{tabular}
\end{flushleft}
\newpage
\section{Gasthermometer}
\subsection{Aufgabenstellung}
Wir zeigen die Gültigkeit des Boyle-Mariotte'schen Gesetzes (\ref{boyle-mariotte}) und bestimmen den Spannungskoeffizienten der Luft $\beta$ und die absolute Temperatur $T_0$ bei $0^\circ C$.
\subsection{Grundlagen}
\subsection{Versuchsaufbau und Methoden}
\subsection{Durchführung}
\subsection{Ergebnisse}
\subsection{Diskussion}

\newpage
\section{Bestimmung des Adiabatenexponenten der Luft nach Rüchardt}
\subsection{Aufgabenstellung}
\subsection{Grundlagen}
\subsection{Versuchsaufbau und Methoden}
\subsection{Durchführung}
\subsection{Ergebnisse}
\subsection{Diskussion}

\newpage
\section{Dampfdichtebestimmung nach Viktor Meyer}
\subsection{Aufgabenstellung}
\subsection{Grundlagen}
\subsection{Versuchsaufbau und Methoden}
\subsection{Durchführung}
\subsection{Ergebnisse}
\subsection{Diskussion}
\end{document}
