\documentclass{article}
\usepackage{amsmath}
\usepackage[utf8]{inputenc}
\usepackage{float}
\usepackage{epsfig,graphicx}
\usepackage{xcolor,import}
\usepackage[german]{babel}
\usepackage{textcomp}
\usepackage{mathtools}

\begin{document}
\thispagestyle{empty}
			\begin{center}
			\Large{Fakultät für Physik}\\
			\end{center}
\begin{verbatim}


\end{verbatim}
							%Eintrag des Wintersemesters
			\begin{center}
			\textbf{\LARGE WINTERSEMESTER 2014/15}
			\end{center}
\begin{verbatim}


\end{verbatim}
			\begin{center}
			\textbf{\LARGE{Physikalisches Praktikum 1}}
			\end{center}
\begin{verbatim}




\end{verbatim}

			\begin{center}
			\textbf{\LARGE{PROTOKOLL}}
			\end{center}
			
\begin{verbatim}





\end{verbatim}

			\begin{flushleft}
			\textbf{\Large{Experiment (Nr., Titel):}}\\
							%Experiment Nr. und Titel statt den Punkten eintragen
			\LARGE{5. Gasthermometer, Adiabatenexponent (Rüchardt), Dampfdichte nach Viktor Meyer}	
			\end{flushleft}

\begin{verbatim}

\end{verbatim}	
							%Eintragen des Abgabedatums, oder des Erstelldatums des Protokolls
			\begin{flushleft}
			\textbf{\Large{Datum: 14.11}}\Large{.2014}
			\end{flushleft}
			
\begin{verbatim}
\end{verbatim}
							%Namen der Protokollschreiber
		\begin{flushleft}
			\textbf{\Large{Namen:}} \Large{Veronika Bachleitner, Erik Grafendorfer}
			\end{flushleft}

\begin{verbatim}


\end{verbatim}
							%Kurstag und Gruppennummer, zb. Fr/5
			\begin{flushleft}
			\textbf{\Large{Kurstag/Gruppe:}} \Large{Fr/1}
			\end{flushleft}

\begin{verbatim}






\end{verbatim}
							%Name des Betreuers, das Praktikum betreute.
			\begin{flushleft}
			\LARGE{\textbf{Betreuer:}}	\Large{SETMAN}	
			\end{flushleft}
\newpage
\section{Allgemeine Grundlagen}
Das Ideale Gas ist ein Modell, bei dem nur Wechselwirkungen durch Stöße der Teilchen untereinander und mit den Wänden angenommen werden.
\textbf{Gleichung des Idealen Gases:}
\begin{equation}
\label{gasgleichung}
pV=nRT
\end{equation}
Experimentell sind Ideale Gase solche, für die in guter Näherung das\\ \textbf{Boyle-Mariotte'sche Gesetz:}
\begin{equation}
\label{boyle-mariotte}
pV=const
\end{equation}
und das \textbf{Gay-Lussac'sche Gesetz:}
\begin{equation}
\label{gay-lussac}
p(t_C)=p(0)(1+\gamma_p t_C)
\end{equation}
erfüllt sind. (Aus \textit{Wagner, Reischl, Steiner: Einführung in die Physik})\\
\\
Hier ist $n$ die Anzahl der Mole der vorliegenden Substanzmenge und weiters
\begin{flushleft}
\begin{tabular}{|l|l|}
\hline Druck & $[p]=Pa$\\
\hline Volumen & $[V]=m^3$\\
\hline Gaskonstante & $R=8.3143 J K^{-1} mol^{-1}$, $R=k_B N_A$\\
\hline Boltzmannkonstante & $k_B=1.3806488*10^{23}J/K$\\
\hline Avogadro'sche oder &\\
Loschmidt'sche Zahl & $N_A=6.02214179*10^{23}$\\
\hline Absolute Temperatur & $[T]=^\circ C$\\
\hline
\end{tabular}
\end{flushleft}
\textbf{Definitionen\footnote{übernommen aus dem Anleitungstext}:}\\
\textbf{1 Mol} = Anzahl von Partikeln, die gleich groß ist wie die Anzahl der Atome in 12g des Isotops $^{12}C$, entspricht $N_A$. \\
\textbf{Molare Masse} = Masse eines Mols in g\\
\textbf{Molekularmasse} = Masse eines Moleküls, ausgedrückt in atomaren Masseneinheiten. \\
1 atomare Masseneinheit = $\frac{1}{12}$ eines $^{12}C$-Atoms.\\
1 amu = $\frac{1}{N_A}$=$1.6605 10^{-27}kg$\\
\textbf{Molekülmasse} = Masse eines Moleküls in g\\
\newpage
\section{Gasthermometer}
\subsection{Aufgabenstellung}
Wir zeigen die Gültigkeit des Boyle-Mariotte'schen Gesetzes (\ref{boyle-mariotte}) und bestimmen den Spannungskoeffizienten der Luft $\beta$ und die absolute Temperatur $T_0$ bei $0^\circ C$.\\
Dabei ermitteln wir auch den Zusammenhang zwischen Druck, Volumen und Temperatur der eingeschlossenen Luftmenge.
\subsection{Grundlagen}
\subsection{Versuchsaufbau und Methoden}
Wir halten die Temperatur konstant (Raumtemperatur) und ändern schrittweise h. Wir lesen dabei die zugehörigen Längen l der Luftsäule ab und multiplizieren diese mit den Gesamtdrücken p. 
$$pV=const \Rightarrow pl=const$$
\subsection{Ergebnisse}
Nachweis des Boyle-Mariotte'schen Gesetzes:\\

$$p_{\vartheta}=p_0 (1+\beta \vartheta)$$
$$\beta=\frac{1}{p_0}\frac{\Delta p}{\Delta \vartheta}=\frac{p_{\vartheta}-p_0}{p_0(\vartheta - \vartheta_0)}$$
$$T_0=$$
\\
Theoretischer Wert für ideale Gase:\\
$$\beta=\frac{1}{T_0}=\frac{1}{273.15K}$$

unsere p*V Werte
178953
17814.8
179628
178318
177024
177830
177328

Atmosphärischer Druck: $P_a=994.5mbar$\\
\\
Messung von l~V, h~$\Delta p$
\begin{center}
\begin{tabular}{r|l|l}
l in mm & h in mmHg & $\Delta p$ in mbar\\
\hline
221 & 62 & 82.46\\
200 & 143 & 190.19\\
186 & 218 & 289.94\\
173 & 283 & 376.39\\
162 & 345 & 458.85\\
154 & 407 & 541.31\\
139 & 528 & 702.24\\
\end{tabular}
\end{center}

2.
\subsection{Diskussion}

\newpage
\section{Bestimmung des Adiabatenexponenten der Luft nach Rüchardt}
\subsection{Aufgabenstellung}
Wir bestimmen den Adiabatenexponenten der Luft mit der Methode von Rüchardt.
\subsection{Grundlagen}
Der Adiabatenexponent ist die Hochzahl in folgender Gleichung:\\
$$p=const./V^{\kappa}$$
Er kann mit der Methode von Rückhardt bestimmt werden. 

\subsection{Versuchsaufbau und Methoden}
\subsection{Durchführung}
\subsection{Ergebnisse}
15 Messungen zu je 5 Schwingungen.\\
Masse der Kugel. (Sartoriuswaage)\\
Adiabatenexponent:
$$\kappa=(\frac{2\pi}{T})^2 \frac{mV_0}{q^2p}$$
$$p=p_0+\frac{mg}{q}$$

$$\boxed{\kappa=1.375093} $$
\subsection{Diskussion}

\newpage
\section{Dampfdichtebestimmung nach Viktor Meyer}
\subsection{Aufgabenstellung}
Wir bestimmen die Dampfdichte $\alpha$ und die Molekularmasse M einer Probesubstanz. 
\subsection{Grundlagen}
\textbf{Dampfdichte}: 
\begin{equation}
\label{dampfdichte_norm}
\alpha=\frac{\rho_{Gas}}{\rho_{L,N}}
\end{equation}
bei Normalbedingungen ($0^\circ C$, 1.01325 bar), wobei $\rho_{Gas}$ die Dichte des Gases, $\rho_{L,N}$ die Dichte der Luft bei Normalbedingungen.\\
\\
Umrechnung für ideale Gase, falls $T\neq 0$:
\begin{equation}
\frac{V_Dp}{T}=\frac{V_Np_N}{T_N}
\end{equation}
wobei $V_D$, $p$, $T$ bei Messbedingungen, $V_N$, $p_N$, $T_N$ bei Normalbedingungen.\\
\\
\textbf{Relative Dampfdichte:}
\begin{equation}
\label{dampfdichte_rel}
\alpha=\frac{\rho_N}{\rho_{L,N}}=\frac{\rho_D}{\rho_{L,N}}\frac{p_NT}{pT_N}
\end{equation}

\subsection{Versuchsaufbau und Methoden}
\subsection{Durchführung}
\subsection{Ergebnisse}

Masse des Kölbchens mit Halterung ohne Flüssigkeit: 7.4899g\\
Dichte der Luft bei Normalbedingungen ($0^\circ C$, 1.01325 bar):\\
$\rho_{L,N}=1.2931kg/m^3$

Setzen wir in (\ref{dampfdichte_rel}) ein, erhalten wir für die Dampfdichte:\\
%\alpha=\frac{\rho_N}{\rho_{L,N}}=\frac{\rho_D}{\rho_{L,N}}\frac{p_NT}{pT_N}\\
\\
Daraus die Molekularmasse:\\
$$M_{Gas}=\alpha*M_L=\alpha*28.98$$

\subsection{Diskussion}
\end{document}
