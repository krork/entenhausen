\documentclass{article}
\usepackage{amsmath}
\usepackage[utf8]{inputenc}
\usepackage{float}
\usepackage{epsfig,graphicx}
\usepackage{xcolor,import}
\usepackage[german]{babel}
\usepackage{textcomp}
\usepackage{mathtools}

\begin{document}
\thispagestyle{empty}
			\begin{center}
			\Large{Fakultät für Physik}\\
			\end{center}
\begin{verbatim}


\end{verbatim}
							%Eintrag des Wintersemesters
			\begin{center}
			\textbf{\LARGE WINTERSEMESTER 2014/15}
			\end{center}
\begin{verbatim}


\end{verbatim}
			\begin{center}
			\textbf{\LARGE{Physikalisches Praktikum 1}}
			\end{center}
\begin{verbatim}




\end{verbatim}

			\begin{center}
			\textbf{\LARGE{PROTOKOLL}}
			\end{center}
			
\begin{verbatim}





\end{verbatim}

			\begin{flushleft}
			\textbf{\Large{Experiment Nr. 10:}} \Large{Wechselstrom I:
Temperaturabhängigkeit des elektrischen Widerstandes,
Transformator}\\
							%Experiment Nr. und Titel statt den Punkten eintragen
			\LARGE{}	
			\end{flushleft}

\begin{verbatim}

\end{verbatim}	
							%Eintragen des Abgabedatums, oder des Erstelldatums des Protokolls
			\begin{flushleft}
			\textbf{\Large{Datum:}} \Large{19.12.2014}
			\end{flushleft}
			
\begin{verbatim}
\end{verbatim}
							%Namen der Protokollschreiber
		\begin{flushleft}
			\textbf{\Large{Namen:}} \Large{Veronika Bachleitner, Erik Grafendorfer}
			\end{flushleft}

\begin{verbatim}


\end{verbatim}
							%Kurstag und Gruppennummer, zb. Fr/5
			\begin{flushleft}
			\textbf{\Large{Kurstag/Gruppe:}} \Large{Fr/1}
			\end{flushleft}

\begin{verbatim}






\end{verbatim}
							%Name des Betreuers, das Praktikum betreute.
			\begin{flushleft}
			\LARGE{\textbf{Betreuer:}}	\Large{KLEPP}	
			\end{flushleft}
\newpage	

\section{Temperaturabhängigkeit des elektrischen Widerstandes}

\subsection{Aufgabenstellung}
Wir bestimmen Eigenschaften verschiedener eines Metalls, eines Halbleiters und eines elektrolytischen Leiters durch Messen ihrer Stromleitungseigenschaften unter verschiedenen Temperaturen.
\subsection{Grundlagen}
\subsubsection*{Metallischer Leiter}
Der temperaturabhängige Widerstand R(T) eines Metallischen Leiters verläuft bei mittleren Temperaturen näherungsweise linear:
\begin{equation}
\label{MetallWiderstand}
R(T) = R_{T0} (1+ \alpha (T-T_0))
\end{equation}
Wobei $ R_{T0}$ der Ausgangs- bzw. Referenzwiderstand ist,
T die Temperatur, und 
$T_0$ die Ausgangs- bzw. Referenztemperatur. 
$\alpha$ ist der Temperaturkoeffizient.\\

\subsubsection*{Halbleiter}
Der Widerstand von Halbleitern bei höheren Temperaturen kann mit einer Exponentialfunktion beschrieben werden.

\begin{equation}
\label{HalbleiterWiderstand}
R(T)=R_{T0}e^{-b(\frac{1}{T_0}-\frac{1}{T})}
\end{equation}

\subsubsection*{Elektrolytischer Leiter}

Die Viskosität $\eta$ einer Flüßigkeit sinkt mit steigender Temperatur - dabei nimmt die Beweglichkeit $\mu$ der Ladungsträger zu, also steigt die Leitfähigkeit $\sigma$.

\begin{equation}
\label{Viskositaet}
\eta = \eta_0 \cdot e^{\frac{E_A}{k_B \cdot T}}
\end{equation}
$E_A$ bezeichnet die Aktivierungsenergie.
\begin{equation}
\label{LeitfaehigkeitElektrolyt}
\sigma  \propto \mu \propto \frac{1}{\eta} \propto e^{-\frac{E_A}{k_B \cdot T}}
\end{equation}

Die Leitfähigkeit $\sigma$ ist indirekt proportional zum Widerstand, also sinkt dieser bei steigender Temperatur.

\begin{equation}
\label{WiderstandElektrolyt}
R(T) \propto \frac{1}{\sigma} \propto e^{\frac{E_A}{k_B \cdot T}}
\end{equation}

\subsection{Versuchsaufbau und Methoden}
\subsubsection*{Metall}
Beim metallischen Leiter formen wir (\ref{MetallWiderstand}) um zu 
\begin{equation}
\label{alpha-metall}
\frac{R(T)}{R_0}-1=\alpha(T-T_0)
\end{equation}
Durch eine lineare Regression können wir den Temperaturkoeffizienten $\alpha$ ermitteln.
\subsubsection*{Halbleiter}
Zur Bestimmung der Lückenenergie $E_g$ wird (\ref{HalbleiterWiderstand}) umgeformt und wir können daraus durch eine lineare Regression den Faktor b ermitteln.

\begin{equation}
\label{LueckenEnergieEg}
f(\frac{1}{T})=\ln\frac{R(T)}{R_{T0}}=b\cdot \frac{1}{T}+d
\end{equation}

\begin{equation}
\label{b-halbleiter}
b=\frac{E_g}{2k_B}
\end{equation}
Mit $k_{B}=8.616*10^{-5}eV/K$ erhalten wir $E_g$.

\subsubsection*{Elektrolyt}
Analog zum Halbleiter ermitteln wir eine Regressionsgerade, deren Steigung die Aktivierungsenergie $E_A$ beschreibt. Wir können die Gleichung (\ref{LueckenEnergieEg}) direkt verwenden. Nur b lautet jetzt
\begin{equation}
\label{b-elektrolyt}
b=\frac{E_A}{k_B}
\end{equation}

\subsection{Durchführung}
\subsection{Ergebnisse}
0 < EA < 20kJ/mol (Sollwert)
\subsection{Diskussion}

\section{Transformator}

\subsection{Aufgabenstellung}
\subsection{Grundlagen}
\subsection{Versuchsaufbau und Methoden}
\subsection{Durchführung}
\subsection{Ergebnisse}
\subsection{Diskussion}
\end{document}
